\documentclass [a4paper, 11pt] {article}

\newcommand\seancetitle{Lignes de transmissions}
\newcommand\seancenumber{2}
%\def\AvecSolutions{}  % Commenter pour ne pas afficher les solutions

\usepackage[utf8]{inputenc}  
\usepackage[T1]{fontenc}  
\usepackage{lmodern}
\usepackage[french]{babel}
\usepackage{fancybox}
\usepackage{listings}
\usepackage{color}
\usepackage{tikz}
\usetikzlibrary{babel}
\usetikzlibrary{decorations.markings}
\usepackage{pgfplots}
\pgfplotsset{compat=1.18}
\pgfplotsset{samples=200}
\usepackage{graphicx,subfigure}
\usepackage[titletoc]{appendix}
\usepackage{float} % figures flottantes 
\usepackage{here} % figures flottantes
\usepackage{url}
\usepackage{enumitem}
\setlist[itemize]{label={$\bullet$}}
%\setlist[enumerate]{noitemsep, nolistsep}
\usepackage{xcolor}
\usepackage[colorlinks=true]{hyperref}
\usepackage{tabularx}
\usepackage{minted}
\usepackage{amsmath}
\usepackage[skins,breakable]{tcolorbox}
\usepackage{verbatim}
\usepackage[europeanresistors,siunitx]{circuitikz}
\usepackage{multicol}
\usepackage{physics}
\usepackage[outline]{contour} % glow around text
%\usetikzlibrary{intersections}
%\usetikzlibrary{decorations.markings}
\usetikzlibrary{angles,quotes} % for pic
\usetikzlibrary{bending} % for arrow head angle
\contourlength{1.0pt}
\usetikzlibrary{3d}
\usetikzlibrary{trees}
\usepackage{dirtytalk}

% --------------------------------------------------------------
% Title
% --------------------------------------------------------------
\makeatletter
\newcommand\maintitle[1]{
    \quitvmode
    \hb@xt@\linewidth{
        \dimen@=1ex
        \advance\dimen@-2pt
        \leaders\hrule \@height1ex \@depth-\dimen@\hfill
        \enskip
        \textbf{#1}
        \enskip
        \leaders\hrule \@height1ex \@depth-\dimen@\hfill
    }
}
\makeatother

\newcommand{\makeseancetitle}{
\begin{center}
    \Large
    \centering
    \maintitle{LELEC1930 - Introduction aux télécommunications}\\
    \textsc{\textbf{Séance \seancenumber{} - \seancetitle{}}}\\
    \vspace{0.1cm}
    \normalsize
    Prof. : Jérôme Louveaux \hfill Assist. : Jérome Eertmans\\
   \noindent\hrulefill
\end{center}
}

\newcommand{\makerappeltitle}{
\begin{center}
    \Large
    \centering
    \maintitle{LELEC1930 - Introduction aux télécommunications}\\
    \textsc{\textbf{Rappel - \seancetitle{}}}\\
    \vspace{0.1cm}
    \normalsize
    Prof. : Jérôme Louveaux \hfill Assist. : Jérome Eertmans\\
   \noindent\hrulefill
\end{center}
}

\usepackage{fancyhdr}
\fancypagestyle{firstpage}
{
   \cfoot{Page \thepage}
}
\fancypagestyle{nextpages}
{
    \lhead{Séance \seancenumber{}}
    \chead{\seancetitle}
    \rhead{LELEC1930}
    \cfoot{Page \thepage}
}
\fancypagestyle{rappelnextpages}
{
    \lhead{Rappel}
    \chead{\seancetitle}
    \rhead{LELEC1930}
    \cfoot{Page \thepage}
}

\setlength{\headheight}{13.59999pt}

% --------------------------------------------------------------
% Some parameters
% --------------------------------------------------------------
\oddsidemargin =0 mm
\topmargin = -10 mm
\footskip = 20mm
\textheight = 240 mm 
\textwidth = 160mm

% --------------------------------------------------------------
% Exercice environments
% --------------------------------------------------------------
\newcounter{exercice}

\definecolor{exercice_color}{RGB}{21,76,121}
\definecolor{exercice_color_fill}{RGB}{252,248,227}

\newcommand{\theexerciceref}{No reference}

\makeatletter
\newenvironment{exercice}[2][\texorpdfstring{\unskip}{}]
{
\refstepcounter{exercice}
\def\@currentlabel{{#2}}
\label{ref-exercice-\theexercice}
\addcontentsline{toc}{subsubsection}{{#2} #1}
\noindent
\flushleft
\begin{tikzpicture}
    \draw[very thick,exercice_color] (0,0) -- ++(0,+7.5pt)
    -- ++(\textwidth,0) node[midway,above] {\textbf{Exercice #2 : #1}}
    -- ++(0,-7.5pt);
\end{tikzpicture}
\vspace{-.3cm}
\begin{tcolorbox}[
    blanker,
    width=\textwidth,
    breakable]
}
{   \end{tcolorbox}
\vspace{-.4cm}
\flushleft
\begin{tikzpicture}
    \draw[very thick,exercice_color] (0,0) -- ++(0,-7.5pt)
    -- ++(\textwidth,0)
    -- ++(0,+7.5pt);
\end{tikzpicture}
}
\makeatother

\ifcsname AvecSolutions\endcsname
\newenvironment{reponse}
{
\noindent
\begin{tcolorbox}[
    colframe=exercice_color,
    colback=exercice_color_fill,
    coltitle=exercice_color_fill,  
    title=\centering\textbf{{\hypersetup{allcolors=white} Réponse à l'exercice \ref{ref-exercice-\theexercice} :}},
    breakable,
    width=\textwidth]
}
{   \end{tcolorbox}
}
\else
\newenvironment{reponse}{\comment}{\endcomment}
\fi


% --------------------------------------------------------------
% Code environments
% --------------------------------------------------------------
\usemintedstyle{borland}
\providecommand*{\listingautorefname}{Listing}

\newenvironment{python}
{\VerbatimEnvironment
\begin{minted}[
linenos,
% fontfamily=courier,
fontsize=\normalsize,
xleftmargin=21pt,
]{python}}
{\end{minted}}

\newcommand\py[1]{\mintinline{python}{#1}}

\newcommand\la[1]{\mintinline{latex}{#1}}

% Vertical line in matrices

\makeatletter
\renewcommand*\env@matrix[1][*\c@MaxMatrixCols c]{%
  \hskip -\arraycolsep
  \let\@ifnextchar\new@ifnextchar
  \array{#1}}
\makeatother

% Double underline
\def\doubleunderline#1{\underline{\underline{#1}}}


\begin{document}

    \makeseancetitle
    \thispagestyle{firstpage}
    
    \part*{Rappel}
    
    Une ligne de transmission peut être modélisée comme sur la \autoref{fig:tline}, dont les paramètres sont décrits dans le \autoref{tab:tline}. Grâce à ce modèle, il est possible de calculer l'impédance de la ligne, $Z_0$, et l'exposant de propagation, $\gamma$,
    
    \vspace{.5cm}
    
    \noindent
    \begin{minipage}{.5\linewidth}
        \begin{equation}\label{eq:impedance}
            Z_0 = \sqrt{\frac{R' + j \omega L'}{G' + j \omega C'}},
        \end{equation}
    \end{minipage}%
    \begin{minipage}{.5\linewidth}
        \begin{equation}
            \gamma = \sqrt{(R' + j \omega L')(G' + j \omega C')},
        \end{equation}
    \end{minipage}
    
    \vspace{.5cm}
    
    \noindent
    avec $\omega$ la pulsation en $\si{\radian\per\second}$. \textbf{Attention} à bien travailler en complexe, car $Z_0$ et $\gamma$ peuvent être complexes. La partie réelle de $\gamma$, notée $\alpha$, est la constante d'atténuation et la partie imaginaire, notée $\beta$, est la constante de phase. Finalement, la vitesse de propagation de l'onde, $v_p$, aussi appelée vitesse de phase, peut s'obtenir :
    
    \begin{equation}
        v_p = \frac{\omega}{\beta}.
    \end{equation}
    
    La constante d'atténuation, $\alpha$, permet de lier la tension d'entrée $V(z,t)$ avec la tension à la sortie $V(z+dz,t)$, après une distance $dz$ donc :
    
    \begin{equation}
        |V(z+dz,t)| = e^{-\alpha dz} |V(z,t)|.
    \end{equation}
    
    \begin{figure}[H]
        \centering
        \begin{circuitikz}
            \draw (0,0) to [short, *-] (8,0)
            to [TL, l_=$G'dz$] (8,4)
            to [short] (5,4)
            (8,4) to [short, -*, i={$i(z+dz,t)$}] (10,4)
            to [open, v^>={$v(z+dz,t)$}] (10,0)
            to [short, *-] (5,0)
            (0,0) to [open, v^<={$v(z, t)$}] (0,4)
            to [short, *-,i={$i(z,t)$}] (1,4)
            to [R, l=$R'dz$] (3,4)
            to [L, l=$L'dz$] (5,4)
            to [C, l=$C'dz$] (5,0);
        \end{circuitikz}
        \caption{Ligne de transmission.}
        \label{fig:tline}
    \end{figure}
    
    \begin{table}[H]
        \centering
        \begin{tabular}{lcl}
            Description & Symbole & Unités \\
            \hline
            Résistance linéique & $R'$ & \si{\ohm\per\meter} \\
            Conductance linéique & $G'$ & \si{\siemens\per\meter} \\
            Inductance linéique & $L'$ & \si{\henry\per\meter} \\
            Capacité linéique & $C'$ & \si{\farad\per\meter}
            
        \end{tabular}
        \caption{Paramètres du modèle de ligne de transmission.}
        \label{tab:tline}
    \end{table}
    
    \pagebreak
    \pagestyle{nextpages}
    
    Dans le cas des télécommunications sans fil, les pertes par câble ne sont pas les seules pertes à prendre en compte (\autoref{fig:radio_losses}). La puissance du signal reçu à l'antenne réceptrice (RX) est une fonction de la puissance isotrope rayonnée équivalente (EIRP en anglais) et de l'affaiblissement de propagation (path loss). L'EIRP se calcule de la manière suivante :
    \begin{equation}
        EIRP\Big|_{\si{\decibel\meter}} = P_t\Big|_{\si{\decibel\meter}} + G_t\Big|_{\si{\decibel}} - L_c\Big|_{\si{\decibel}},
    \end{equation}
    où $P_t$ est la puissance transmise par l'émetteur, $G_t$ est le gain de l'antenne émettrice et $L_c$ sont les pertes liées au câble reliant l'émetteur et l'antenne.
    
    Ensuite, l'affaiblissement de propagation dans un cas d'espace vide (free space), c.-à-d. sans obstacle, s'obtient comme suit :
    \begin{equation}
        L_{FS}\Big|_{\si{\decibel}} \approx 32.4 + 20 \log(f\Big|_{\si{\mega\hertz}}) + 20 \log(D\Big|_{\si{\kilo\meter}}),
    \end{equation}
    avec $f$ la fréquence (en \si{\mega\hertz}) et $D$ la distance inter-antennes (en \si{\kilo\meter}). La fonction $\log(x)$ désigne le logarithme en base 10, $\log_{10}(x)$.
    
    De là, on peut finalement calculer le niveau de puissance reçu à l'antenne réceptrice (RSL) :
    
    \begin{equation}
        RSL\Big|_{\si{\decibel\meter}} = EIRP\Big|_{\si{\decibel\meter}} + G_r\Big|_{\si{\decibel}} - L_{FS}\Big|_{\si{\decibel}},
    \end{equation}
    avec $G_r$ le gain de l'antenne réceptrice.
    
    \begin{figure}[H]
            \centering
            \begin{tikzpicture}
                \node[circle,draw,label=above:{TX}] (TX) at (0,0) {};
                \node[circle,draw,label=above:{RX}] (RX) at (12,0) {};
                \path (TX) -- ++(0,-4) node[draw,rectangle,rounded corners] (transmetteur) {Émetteur};
                \draw[->,thick,dashed] (transmetteur) -- (TX) node[midway,above,sloped,align=center] {Ligne de\\transmission};
                \draw (RX) -- (RX |- transmetteur.north) coordinate (bottom);
                \draw (bottom) ++(-.5,0) -- ++(1,0);

                \draw[|<->|] (TX.center |-, -4.7) -- (RX.center |-, -4.7) node[midway,above,sloped] {Distance inter-antennes};
                
                \foreach \i in {1,...,3} {
                    \draw (TX) ++(30:\i) arc (30:-30:\i);
                }
            \end{tikzpicture}
            \caption{Calcul de pertes lors de télécommunications.}
            \label{fig:radio_losses}
        \end{figure}
    
    \pagebreak
    \part*{Exercices}
    
    \begin{exercice}[Câble vidéo]{1}
        Considérez un câble coaxial RG218U, fréquemment utilisé pour la transmission vidéo dans les ultra hautes fréquences (UHF). Les valeurs des paramètres primaires sont données dans le \autoref{tab:exo1}. La conductance linéique $G'$ est supposée négligeable.
        
        \begin{table}[H]
            \centering
            \begin{tabular}{cl}
                Symbole & Valeur \\
                \hline
                $R'$ & \SI{0.368}{\ohm\per\meter} \\
                $L'$ & \SI{250}{\nano\henry\per\meter} \\
                $C'$ & \SI{102.4}{\pico\farad\per\meter}
            \end{tabular}
            \caption{Paramètres du modèle de ligne de transmission.}
            \label{tab:exo1}
        \end{table}
        
        \begin{enumerate}
            \item Calculez l'impédance caractéristique du câble, la constance d'atténuation et la vitesse de phase pour un signal à \SI{100}{\mega\hertz}.
            \item En supposant une puissance à l'entrée de \SI{66}{\micro\watt}, déterminez la longueur maximum de câble pour que la puissance à la sortie soit au moins égale à \SI{0.42}{\micro\watt}.
        \end{enumerate}
    \end{exercice}
    
    \begin{reponse}
        Une simple application des formules du rappel permet de répondre aux questions :
        
        \begin{enumerate}
            \item \begin{enumerate}
                \item $Z_0=49.41-0.06j$, $|Z_0|=\SI{49.41}{\ohm}$
                \item $\alpha=\SI{0.0037}{\per\meter}$
                \item $v_p=\SI{197642218}{\meter\per\second}$
            \end{enumerate}
            \item Comme la puissance est proportionnelle au carré de la tension (loi d'Ohm), on peut écrire :
            \begin{equation}
                \frac{\SI{0.42}{\micro\watt}}{\SI{66}{\micro\watt}} = \frac{|V(z+dz,t)|^2}{|V(z,t)|^2} = e^{-2\alpha dz}.
            \end{equation}
            Ensuite, en isolant la longueur de câble $dz$ :
            \begin{equation}
                dz = \frac{1}{-2\alpha}\ln\left(\frac{\SI{0.42}{\micro\watt}}{\SI{66}{\micro\watt}}\right) = \frac{1}{2\alpha}\ln\left(\frac{\SI{66}{\micro\watt}}{\SI{0.42}{\micro\watt}}\right) = \SI{679}{\meter}.
            \end{equation}
        \end{enumerate}
    \end{reponse}
    
    \begin{exercice}[Adaptation d'impédance]{2}
        En pratique, on souhaite souvent avoir une impédance de ligne bien précise afin que les câbles utilisés puissent être compatibles avec un maximum d'appareils différents. Ici, on souhaite avoir une impédance de ligne $|Z_0|=\SI{50}{\ohm}$.
        
        \begin{enumerate}
            \item Dans un premier temps, calculez la fréquence qui permettrait d'atteindre l'impédance voulue (sur base des paramètres donnés dans le \autoref{tab:exo1}).
            \item Dans un second temps, on va plutôt changer la résistance linéique du câble\footnote{En pratique, on ne peut pas changer la composition du câble à notre guise. On va donc plutôt rajouter une résistance en fin de ligne pour compenser la différence entre l'impédance réelle et celle voulue.}. Calculez la valeur de résistance linéique $R'$ qu'il faudrait utiliser pour avoir l'impédance recherchée (pour le signal à \SI{100}{\mega\hertz}).
            \item Est-il possible de rajouter une conductance linéique afin d'augmenter l'impédance de ligne ?
        \end{enumerate}
    \end{exercice}
    
    \begin{reponse}
        En prenant le module de l'\autoref{eq:impedance} et en mettant tout au carré, on obtient :
        \begin{equation}
            50^2 =\left|\frac{R' + j \omega L'}{G' + j \omega C'}\right|.
        \end{equation}
        Ensuite, on peut résoudre (analytiquement, avec la calculette ou via \py{Python}) l'équation pour trouver $f=\frac{\omega}{2\pi}$ ou $R'$:
        \begin{equation}
            2500 =\frac{\sqrt{R'^2 + \omega^2 L'^2}}{\sqrt{G'^2 + \omega^2 C'^2}} \Rightarrow 2500^2 =\frac{R'^2 + \omega^2 L'^2}{G'^2 + \omega^2 C'^2}.
        \end{equation}
        \begin{enumerate}
            \item On obtient environ \SI{1.06}{\mega\hertz} comme fréquence nécessaire pour avoir une impédance de \SI{50}{\ohm}.
            \item On calcule qu'il faudrait augmenter la résistance linéique du câble à \SI{34.76}{\ohm\per\meter}.
            \item Non, rajouter l'effet de la conductance linéique ne peut que réduire l'impédance de ligne.
        \end{enumerate}
    \end{reponse}
    
    \begin{exercice}[Lien radio]{3}
    
        Considérez le lien radio illustré sur la \autoref{fig:radio_link}, avec les données suivantes :
        
        \begin{enumerate}
            \item L'émetteur travaille à une fréquence de \SI{1}{\giga\hertz}.
            \item L'émetteur et son antenne sont séparés par un câble coaxial RG9 de \SI{15}{\meter}. L'atténuation de ce dernier est d'environ \SI{0.23}{\decibel\per\meter}.
            \item La distance entre l'antenne émettrice (TX) et l'antenne réceptrice (RX) est de \SI{8}{\kilo\meter}.
            \item L'antenne TX possède un gain de \SI{17}{\decibel} et une perte due à la connexion câblée de \SI{1.5}{\decibel}.
            \item L'antenne RX possède un gain de \SI{2}{\decibel} et un seul de sensibilité à \SI{-75}{\decibel\meter}.
        \end{enumerate}
        
        Calculez la puissance minimale requise à l'émetteur pour que le signal soit bien reçu.
    
        \begin{figure}[H]
            \centering
            \begin{tikzpicture}
                \node[circle,draw,label=above:{TX}] (TX) at (0,0) {};
                \node[circle,draw,label=above:{RX}] (RX) at (12,0) {};
                \path (TX) -- ++(0,-2) node[draw,rectangle,rounded corners] (transmetteur) {Émetteur};
                \draw[->,thick,dashed] (transmetteur) -- (TX) node[midway,above,sloped] {RG9};
                \draw (RX) -- (RX |- transmetteur.north) coordinate (bottom);
                \draw (bottom) ++(-.5,0) -- ++(1,0);
                
                \draw[|<->|] (-2, |- transmetteur) -- (-2, |- TX.center) node[midway,above,sloped] {\SI{15}{\meter}};
                \draw[|<->|] (TX.center |-, -3) -- (RX.center |-, -3) node[midway,above,sloped] {\SI{8}{\kilo\meter}};
                
                \foreach \i in {1,...,3} {
                    \draw (TX) ++(30:\i) arc (30:-30:\i);
                }
            \end{tikzpicture}
            \caption{Lien radio.}
            \label{fig:radio_link}
        \end{figure}
    \end{exercice}
    
    \begin{reponse}
        En reprenant ce qui est donné dans le rappel, on peut d'abord calculer l'EIRP :
        \begin{equation}
            EIRP\Big|_{\si{\decibel\meter}} = P_t\Big|_{\si{\decibel\meter}} + 17 - L_c\Big|_{\si{\decibel}}
        \end{equation}
        avec $L_c = 1.5 + 15\cdot 0.23 = \SI{4.95}{\decibel}$.
        
        Ensuite, l'affaiblissement de propagation peut se calculer :
        \begin{equation}
            L_{FS}\Big|_{\si{\decibel}} \approx 32.4 + 20 \log(1000) + 20 \log(8) = \SI{110.5}{\decibel},
        \end{equation}
        
        Pour finir, on peut trouver la puissance minimale à l'émission afin d'obtenir un signal supérieur ou égal à \SI{-75}{\decibel\meter} à l'antenne réceptrice :
        \begin{equation}
            -75 \leq P_t\Big|_{\si{\decibel\meter}} + 17 - 4.95 + 2 - 110.5
        \end{equation}
        
        On trouve $P_t \geq \SI{21.4}{\decibel\meter}$.
        
        
    \end{reponse}
        
   

\end{document}