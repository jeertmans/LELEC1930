\documentclass [a4paper, 11pt] {article}

\newcommand\seancetitle{Transformée de Fourier}
\newcommand\seancenumber{0}
%\def\AvecSolutions{}  % Commenter pour ne pas afficher les solutions

\usepackage[utf8]{inputenc}  
\usepackage[T1]{fontenc}  
\usepackage{lmodern}
\usepackage{amsmath}
\usepackage[french]{babel}
\usepackage{fancybox}
\usepackage{listings}
\usepackage{color}
\usepackage{tikz}
\usetikzlibrary{babel}
\usetikzlibrary{decorations.markings}
\usepackage{pgfplots}
\pgfplotsset{compat=1.17}
\pgfplotsset{samples=200}
\usepackage{graphicx,subfigure}
\usepackage[titletoc]{appendix}
\usepackage{float} % figures flottantes 
\usepackage{here} % figures flottantes
\usepackage{url}
\usepackage{enumitem}
\setlist[itemize]{label={$\bullet$}}
%\setlist[enumerate]{noitemsep, nolistsep}
\usepackage{xcolor}
\usepackage[colorlinks=true]{hyperref}
\usepackage{tabularx}
\usepackage{minted}
\usepackage[skins,breakable]{tcolorbox}
\usepackage{verbatim}
\usepackage[europeanresistors,siunitx]{circuitikz}
\usepackage{multicol}
\usepackage{physics}
\usepackage[outline]{contour} % glow around text
%\usetikzlibrary{intersections}
%\usetikzlibrary{decorations.markings}
\usetikzlibrary{angles,quotes} % for pic
\usetikzlibrary{bending} % for arrow head angle
\contourlength{1.0pt}
\usetikzlibrary{3d}
\usetikzlibrary{trees}
\usepackage{dirtytalk}

% --------------------------------------------------------------
% Title
% --------------------------------------------------------------
\makeatletter
\newcommand\maintitle[1]{
    \quitvmode
    \hb@xt@\linewidth{
        \dimen@=1ex
        \advance\dimen@-2pt
        \leaders\hrule \@height1ex \@depth-\dimen@\hfill
        \enskip
        \textbf{#1}
        \enskip
        \leaders\hrule \@height1ex \@depth-\dimen@\hfill
    }
}
\makeatother

\newcommand{\makeseancetitle}{
\begin{center}
    \Large
    \centering
    \maintitle{LELEC1930 - Introduction aux télécommunications}\\
    \textsc{\textbf{Séance \seancenumber{} - \seancetitle{}}}\\
    \vspace{0.1cm}
    \normalsize
    Prof. : Jérôme Louveaux \hfill Assist. : Jérome Eertmans\\
   \noindent\hrulefill
\end{center}
}

\newcommand{\makerappeltitle}{
\begin{center}
    \Large
    \centering
    \maintitle{LELEC1930 - Introduction aux télécommunications}\\
    \textsc{\textbf{Rappel - \seancetitle{}}}\\
    \vspace{0.1cm}
    \normalsize
    Prof. : Jérôme Louveaux \hfill Assist. : Jérome Eertmans\\
   \noindent\hrulefill
\end{center}
}

\usepackage{fancyhdr}
\fancypagestyle{firstpage}
{
   \cfoot{Page \thepage}
}
\fancypagestyle{nextpages}
{
    \lhead{Séance \seancenumber{}}
    \chead{\seancetitle}
    \rhead{LELEC1930}
    \cfoot{Page \thepage}
}
\fancypagestyle{rappelnextpages}
{
    \lhead{Rappel}
    \chead{\seancetitle}
    \rhead{LELEC1930}
    \cfoot{Page \thepage}
}

\setlength{\headheight}{13.59999pt}

% --------------------------------------------------------------
% Some parameters
% --------------------------------------------------------------
\oddsidemargin =0 mm
\topmargin = -10 mm
\footskip = 20mm
\textheight = 240 mm 
\textwidth = 160mm

% --------------------------------------------------------------
% Exercice environments
% --------------------------------------------------------------
\newcounter{exercice}

\definecolor{exercice_color}{RGB}{21,76,121}
\definecolor{exercice_color_fill}{RGB}{252,248,227}

\newcommand{\theexerciceref}{No reference}

\makeatletter
\newenvironment{exercice}[2][\texorpdfstring{\unskip}{}]
{
\refstepcounter{exercice}
\def\@currentlabel{{#2}}
\label{ref-exercice-\theexercice}
\addcontentsline{toc}{subsubsection}{{#2} #1}
\noindent
\flushleft
\begin{tikzpicture}
    \draw[very thick,exercice_color] (0,0) -- ++(0,+7.5pt)
    -- ++(\textwidth,0) node[midway,above] {\textbf{Exercice #2 : #1}}
    -- ++(0,-7.5pt);
\end{tikzpicture}
\vspace{-.3cm}
\begin{tcolorbox}[
    blanker,
    width=\textwidth,
    breakable]
}
{   \end{tcolorbox}
\vspace{-.4cm}
\flushleft
\begin{tikzpicture}
    \draw[very thick,exercice_color] (0,0) -- ++(0,-7.5pt)
    -- ++(\textwidth,0)
    -- ++(0,+7.5pt);
\end{tikzpicture}
}
\makeatother

\ifcsname AvecSolutions\endcsname
\newenvironment{reponse}
{
\noindent
\begin{tcolorbox}[
    colframe=exercice_color,
    colback=exercice_color_fill,
    coltitle=exercice_color_fill,  
    title=\centering\textbf{{\hypersetup{allcolors=white} Réponse à l'exercice \ref{ref-exercice-\theexercice} :}},
    breakable,
    width=\textwidth]
}
{   \end{tcolorbox}
}
\else
\newenvironment{reponse}{\comment}{\endcomment}
\fi


% --------------------------------------------------------------
% Code environments
% --------------------------------------------------------------
\usemintedstyle{borland}
\providecommand*{\listingautorefname}{Listing}

\newenvironment{python}
{\VerbatimEnvironment
\begin{minted}[
linenos,
% fontfamily=courier,
fontsize=\normalsize,
xleftmargin=21pt,
]{python}}
{\end{minted}}

\newcommand\py[1]{\mintinline{python}{#1}}

\newcommand\la[1]{\mintinline{latex}{#1}}

% Vertical line in matrices

\makeatletter
\renewcommand*\env@matrix[1][*\c@MaxMatrixCols c]{%
  \hskip -\arraycolsep
  \let\@ifnextchar\new@ifnextchar
  \array{#1}}
\makeatother

% Double underline
\def\doubleunderline#1{\underline{\underline{#1}}}


\begin{document}

    \makeseancetitle
    \thispagestyle{firstpage}
    
    \part*{Rappel}
    
    Transformée de Fourier d'un signal $x(t)$ :
    \begin{equation}
        X(\omega) = \int\limits_{-\infty}^{\infty} x(t) e^{-j\omega t} dt
    \end{equation}
    
    Transformée de Fourier inverse :
    \begin{equation}
        x(t) = \frac{1}{2\pi} \int\limits_{-\infty}^{\infty} X(\omega) e^{-j\omega t} d\omega
    \end{equation}
    
    Propriété de la transformée de Fourier : 
    \begin{center}
        si $x(t)$ est réel, alors $X(-\omega)=X^*(\omega)$, 
    \end{center}
    où $^*$ est l'opérateur conjugué, tel que $(a+b\cdot j)^*=a - b\cdot j$, avec $a$ et $b$ des nombres réels.
    
    \vspace{2cm}
    \begin{figure}[H]
        \centering
        \begin{tikzpicture}
        \begin{axis}[
            set layers=standard,
            domain=0:10,
            samples y=1,
            view={40}{20},
            hide axis,
            unit vector ratio*=1 2 1,
            xtick=\empty, ytick=\empty, ztick=\empty,
            clip=false,
            scale=2,
        ]
        \def\sumcurve{0}
        \pgfplotsinvokeforeach{0.5,1.5,...,5.5}{
            \draw [on layer=background, gray!20] (axis cs:0,#1,0) -- (axis cs:10,#1,0);
            \addplot3 [on layer=main, blue!30, smooth, samples=101]
              (x,#1,{sin(#1*x*(157))/(#1*2)});
            \addplot3 [on layer=axis foreground, very thick, blue,ycomb, samples=2]
              (10.5,#1,{1/(#1*2)});
            \xdef\sumcurve{\sumcurve + sin(#1*x*(157))/(#1*2)}
        }
        \addplot3 [red, samples=200] (x,0,{\sumcurve});
        \draw [on layer=axis foreground]  (axis cs:0,0,0) -- (axis cs:10,0,0);
        \draw (axis cs:10.5,0.25,0) -- (axis cs:10.5,5.5,0);
        \end{axis}
        \end{tikzpicture}
        \caption{Illustration de la transformée de Fourier, repris de \href{https://pgfplots.net/fourier-transform/}{pgfplots.net}.}
        \label{fig:fourier}
    \end{figure}
    
    Pour illustrer un signal en fréquentielle, qui est souvent complexe, on représente souvent son module ($|a+b\cdot j| = \sqrt{a^2 + b^2}$), et parfois sa phase ($\angle a+b\cdot j = \arctan (b/a)$), en fonction de $\omega$.
    
    
    \pagebreak
    \pagestyle{nextpages}
    \part*{Exercices}
    
    \begin{exercice}[Calculs]{1}
        Calculez la transformée de Fourier des signaux suivants. Représentez à chaque fois le signal en temporel et sa transformée de Fourier en fréquentiel.
        
        \begin{itemize}
            \item $x(t)=\left\{\begin{array}{ll}e^{-t}&\mbox{pour }t\geq 0\\0&\mbox{pour }t<0\end{array}\right.$
            \item $x(t)=\left\{\begin{array}{ll}1&\mbox{pour }-T\leq t\leq T\\0&\mbox{ailleurs}\end{array}\right.$
            \item $x(t)=\left\{\begin{array}{ll}1&\mbox{pour } 0\leq t\leq 2T\\0&\mbox{ailleurs}\end{array}\right.$
        \end{itemize}
    
        Ensuite, calculez la transformée de Fourier inverse de
    
        \begin{equation}
        X(\omega)=\left\{\begin{array}{ll}e^{-2\omega}&\mbox{pour } \omega \geq 0\\0&\mbox{pour } \omega<0 \end{array}\right.
        \end{equation}


    \end{exercice}
        
    \begin{reponse}
        En appliquant la définition de la transformée de Fourier, on trouve :
        \begin{itemize}
            \item $X(\omega)=\left[\frac{e^{-t(1+j\omega)}}{1+j\omega}\right]_0^\infty=\frac{1}{1+j\omega}$ si $\Re(j\omega)>-1$
            \begin{figure}[H]
                \centering
                \begin{tikzpicture}
                \begin{axis}[
                    xlabel={$\omega$ (\si{\radian\per\second})},  
                    ylabel={$|X(\omega)|$},
                    xmax = 12,
                    ymax = 1.1,
                    domain=-1:10,
                    grid=both,
                ]
                \addplot [blue,no marks,smooth]{1/sqrt(x^2 + 1)};
                \end{axis}
                \end{tikzpicture}
            \end{figure}
            \item $X(\omega)=-\left[\frac{e^{-jt\omega}}{j\omega}\right]_{-T}^T=\frac{2\sin{(T\omega)}}{\omega}$
            \begin{figure}[H]
                \centering
                \begin{tikzpicture}
                \begin{axis}[
                    xlabel={$\omega$ (\si{\radian\per\second})},  
                    ylabel={$|X(\omega)|$},
                    grid=both,
                    %xmax = 1,
                    ymax = 2.2,
                    domain=-1440:1440,
                    xtick={-720,-360,...,720},
                    xticklabels={$-\frac{4\pi}{T}$,$-\frac{2\pi}{T}$,0,$\frac{2\pi}{T}$,$\frac{4\pi}{T}$},
                ]
                \addplot [blue,no marks]{2*sin(x)/(x*pi/180)};
                \end{axis}
                \end{tikzpicture}
            \end{figure}
            \item $X(\omega)=-\left[\frac{e^{-jt\omega}}{j\omega}\right]_{0}^{2T}=\frac{1- e^{-2jT\omega}}{j\omega}$
            \begin{figure}[H]
                \centering
                \begin{tikzpicture}
                \begin{axis}[
                    xlabel={$\omega$ (\si{\radian\per\second})},  
                    ylabel={$|X(\omega)|$},
                    grid=both,
                    %xmax = 1,
                    ymax = 2.2,
                    domain=-1440:1440,
                    xtick={-720,-360,...,720},
                    xticklabels={$-\frac{4\pi}{T}$,$-\frac{2\pi}{T}$,0,$\frac{2\pi}{T}$,$\frac{4\pi}{T}$},
                ]
                \addplot [blue,no marks]{2*abs(sin(x)/(x*pi/180))};
                \end{axis}
                \end{tikzpicture}
            \end{figure}
        \end{itemize}
        
        De même pour la transformée inverse :
        
        \begin{equation}
            x(t) = -\left[\frac{e^{-\omega(jt+2)}}{2\pi(jt+2)}\right]_{0}^{\infty}=\frac{1}{2\pi(jt+2)}
        \end{equation}
    \end{reponse}
    
    \begin{exercice}[Propriétés]{2}
        Par calcul, à partir de la définition de la transformée, montrez les propriétés suivantes :
        
        \begin{itemize}
            \item Linéarité : si $x(t)=x_1(t)+x_2(t)$, alors $X(\omega)=X_1(\omega)+X_2(\omega)$.
            \item Translation : la transformée de $x(t-a)$ est $e^{-j\omega a}X(\omega)$.
            \item Modulation : la transformée de $x(t)\cos(\omega_0 t)$ est $\frac{1}{2}\left(X(\omega-\omega_0)+X(\omega+\omega_0)\right)$.
        \end{itemize}

    \end{exercice}
    
    \begin{reponse}
        \begin{itemize}
            \item Linéarité : en injectant $x(t)$ dans la définition de la transformée, on trouve
            \begin{equation}
                X(\omega)= \int\limits_{-\infty}^\infty (x_1(t)+x_2(t)) e^{-j\omega t} dt = \underbrace{\int\limits_{-\infty}^\infty x_1(t) e^{-j\omega t} dt}_{X_1(\omega)} + \underbrace{\int\limits_{-\infty}^\infty x_2(t) e^{-j\omega t} dt}_{X_2(\omega)} 
            \end{equation}
            \item Translation : en substituant $t-a$ par $s$ (donc $t=a+s$), on retrouve $X(\omega)$ à une constante près
            \begin{equation}
                X(\omega)= \int\limits_{-\infty}^{\infty} x(t-a) e^{-j\omega t} dt = \int\limits_{-\infty}^{\infty} x(s) e^{-j\omega (a+s)} ds =  e^{-j\omega a} \underbrace{\int\limits_{-\infty}^{\infty} x(s) e^{-j\omega s} ds}_{X(\omega)}
            \end{equation}
            \item Modulation : en réécrivant le $\cos(\omega_0 t)=\frac{e^{j\omega_0 t}+e^{-j\omega_0 t}}{2}$, on trouve
            \begin{align}
                X(\omega) &= \int\limits_{-\infty}^\infty x(t)\frac{e^{j\omega_0 t}+e^{-j\omega_0 t}}{2}e^{-j\omega t} dt \\
                &= \frac{1}{2}\left(\underbrace{\int\limits_{-\infty}^\infty x(t) e^{-j(\omega-\omega_0) t} dt}_{X(\omega-\omega_0)} + \underbrace{\int\limits_{-\infty}^\infty x(t) e^{-j(\omega+\omega_0) t} dt}_{X(\omega+\omega_0)} \right)
            \end{align}
        \end{itemize}
    \end{reponse}
    
    \begin{exercice}[Signaux réels]{3}
        \begin{enumerate}
            \item Si $x(t)$ est un signal réel, et $X(2)=1+j$, que vaut $X(-2)$ ? Que peut-on dire sur $X(0)$ ?
            \item Pour chacune des transformées de Fourier ci-dessus, indiquez si elle correspond à un signal réel :            
            \begin{itemize}
                \item $X(\omega)=e^{-\omega}$
                \item $X(\omega)=e^{-|\omega|}$
                \item $X(\omega)=\frac{\sin(\omega)}{\omega}$
                \item $X(\omega)=\left\{\begin{array}{ll}(1+j)&\mbox{pour } -\omega_0 \leq \omega \leq \omega_0\\0&\mbox{ailleurs} \end{array}\right.$
            \end{itemize}
        \end{enumerate}

    \end{exercice}
    
    \begin{reponse}
        \begin{enumerate}
            \item Comme le signal est réel, $X(-2)=X^*(2)=1-j$. Pour ce qui est de $X(0)$, sa partie imaginaire doit obligatoirement être nulle.
            \item Pour vérifier si le signal de base est réel, il suffit de vérifier que $X(-\omega)=X^*(\omega)$:
            \begin{itemize}
                \item Non $\Rightarrow$ $X(-\omega)=e^{\omega}\ne X^*(\omega)=e^{-\omega}$
                \item Oui $\Rightarrow$ $X(-\omega)=e^{-|\omega|} = X^*(\omega)=e^{-|\omega|}$
                \item Oui $\Rightarrow$ $X(-\omega)=\frac{-\sin(\omega)}{-\omega}=X^*(\omega)=\frac{\sin(\omega)}{\omega}$
                \item Non $\Rightarrow$ $-\omega_0 \leq \omega \leq \omega_0$, $X(-\omega)=1+j\ne X^*(\omega)=1-j$
            \end{itemize}
        \end{enumerate}
    \end{reponse}

\end{document}