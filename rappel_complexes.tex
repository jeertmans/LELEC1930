\documentclass [a4paper, 11pt] {article}

\newcommand\seancetitle{Nombres complexes}

\usepackage[utf8]{inputenc}  
\usepackage[T1]{fontenc}  
\usepackage{lmodern}
\usepackage[french]{babel}
\usepackage{fancybox}
\usepackage{listings}
\usepackage{color}
\usepackage{tikz}
\usetikzlibrary{babel}
\usetikzlibrary{decorations.markings}
\usepackage{pgfplots}
\pgfplotsset{compat=1.18}
\pgfplotsset{samples=200}
\usepackage{graphicx,subfigure}
\usepackage[titletoc]{appendix}
\usepackage{float} % figures flottantes 
\usepackage{here} % figures flottantes
\usepackage{url}
\usepackage{enumitem}
\setlist[itemize]{label={$\bullet$}}
%\setlist[enumerate]{noitemsep, nolistsep}
\usepackage{xcolor}
\usepackage[colorlinks=true]{hyperref}
\usepackage{tabularx}
\usepackage{minted}
\usepackage{amsmath}
\usepackage[skins,breakable]{tcolorbox}
\usepackage{verbatim}
\usepackage[europeanresistors,siunitx]{circuitikz}
\usepackage{multicol}
\usepackage{physics}
\usepackage[outline]{contour} % glow around text
%\usetikzlibrary{intersections}
%\usetikzlibrary{decorations.markings}
\usetikzlibrary{angles,quotes} % for pic
\usetikzlibrary{bending} % for arrow head angle
\contourlength{1.0pt}
\usetikzlibrary{3d}
\usetikzlibrary{trees}
\usepackage{dirtytalk}

% --------------------------------------------------------------
% Title
% --------------------------------------------------------------
\makeatletter
\newcommand\maintitle[1]{
    \quitvmode
    \hb@xt@\linewidth{
        \dimen@=1ex
        \advance\dimen@-2pt
        \leaders\hrule \@height1ex \@depth-\dimen@\hfill
        \enskip
        \textbf{#1}
        \enskip
        \leaders\hrule \@height1ex \@depth-\dimen@\hfill
    }
}
\makeatother

\newcommand{\makeseancetitle}{
\begin{center}
    \Large
    \centering
    \maintitle{LELEC1930 - Introduction aux télécommunications}\\
    \textsc{\textbf{Séance \seancenumber{} - \seancetitle{}}}\\
    \vspace{0.1cm}
    \normalsize
    Prof. : Jérôme Louveaux \hfill Assist. : Jérome Eertmans\\
   \noindent\hrulefill
\end{center}
}

\newcommand{\makerappeltitle}{
\begin{center}
    \Large
    \centering
    \maintitle{LELEC1930 - Introduction aux télécommunications}\\
    \textsc{\textbf{Rappel - \seancetitle{}}}\\
    \vspace{0.1cm}
    \normalsize
    Prof. : Jérôme Louveaux \hfill Assist. : Jérome Eertmans\\
   \noindent\hrulefill
\end{center}
}

\usepackage{fancyhdr}
\fancypagestyle{firstpage}
{
   \cfoot{Page \thepage}
}
\fancypagestyle{nextpages}
{
    \lhead{Séance \seancenumber{}}
    \chead{\seancetitle}
    \rhead{LELEC1930}
    \cfoot{Page \thepage}
}
\fancypagestyle{rappelnextpages}
{
    \lhead{Rappel}
    \chead{\seancetitle}
    \rhead{LELEC1930}
    \cfoot{Page \thepage}
}

\setlength{\headheight}{13.59999pt}

% --------------------------------------------------------------
% Some parameters
% --------------------------------------------------------------
\oddsidemargin =0 mm
\topmargin = -10 mm
\footskip = 20mm
\textheight = 240 mm 
\textwidth = 160mm

% --------------------------------------------------------------
% Exercice environments
% --------------------------------------------------------------
\newcounter{exercice}

\definecolor{exercice_color}{RGB}{21,76,121}
\definecolor{exercice_color_fill}{RGB}{252,248,227}

\newcommand{\theexerciceref}{No reference}

\makeatletter
\newenvironment{exercice}[2][\texorpdfstring{\unskip}{}]
{
\refstepcounter{exercice}
\def\@currentlabel{{#2}}
\label{ref-exercice-\theexercice}
\addcontentsline{toc}{subsubsection}{{#2} #1}
\noindent
\flushleft
\begin{tikzpicture}
    \draw[very thick,exercice_color] (0,0) -- ++(0,+7.5pt)
    -- ++(\textwidth,0) node[midway,above] {\textbf{Exercice #2 : #1}}
    -- ++(0,-7.5pt);
\end{tikzpicture}
\vspace{-.3cm}
\begin{tcolorbox}[
    blanker,
    width=\textwidth,
    breakable]
}
{   \end{tcolorbox}
\vspace{-.4cm}
\flushleft
\begin{tikzpicture}
    \draw[very thick,exercice_color] (0,0) -- ++(0,-7.5pt)
    -- ++(\textwidth,0)
    -- ++(0,+7.5pt);
\end{tikzpicture}
}
\makeatother

\ifcsname AvecSolutions\endcsname
\newenvironment{reponse}
{
\noindent
\begin{tcolorbox}[
    colframe=exercice_color,
    colback=exercice_color_fill,
    coltitle=exercice_color_fill,  
    title=\centering\textbf{{\hypersetup{allcolors=white} Réponse à l'exercice \ref{ref-exercice-\theexercice} :}},
    breakable,
    width=\textwidth]
}
{   \end{tcolorbox}
}
\else
\newenvironment{reponse}{\comment}{\endcomment}
\fi


% --------------------------------------------------------------
% Code environments
% --------------------------------------------------------------
\usemintedstyle{borland}
\providecommand*{\listingautorefname}{Listing}

\newenvironment{python}
{\VerbatimEnvironment
\begin{minted}[
linenos,
% fontfamily=courier,
fontsize=\normalsize,
xleftmargin=21pt,
]{python}}
{\end{minted}}

\newcommand\py[1]{\mintinline{python}{#1}}

\newcommand\la[1]{\mintinline{latex}{#1}}

% Vertical line in matrices

\makeatletter
\renewcommand*\env@matrix[1][*\c@MaxMatrixCols c]{%
  \hskip -\arraycolsep
  \let\@ifnextchar\new@ifnextchar
  \array{#1}}
\makeatother

% Double underline
\def\doubleunderline#1{\underline{\underline{#1}}}


\tikzset{>=latex} % for LaTeX arrow head
\usepackage{xcolor}
\colorlet{myblue}{blue!65!black}
\colorlet{mydarkblue}{blue!50!black}
\colorlet{myred}{red!65!black}
\colorlet{mydarkred}{red!40!black}
\colorlet{veccol}{green!70!black}
\colorlet{vcol}{green!70!black}
\colorlet{xcol}{blue!85!black}
%\colorlet{projcol}{xcol!60}
%\colorlet{unitcol}{xcol!60!black!85}
%\colorlet{myred}{red!90!black}
%\colorlet{mypurple}{blue!50!red!80!black!80}
\tikzstyle{vector}=[->,very thick,xcol,line cap=round]
\tikzstyle{xline}=[myblue,very thick]
\tikzstyle{yzp}=[canvas is zy plane at x=0]
\tikzstyle{xzp}=[canvas is xz plane at y=0]
\tikzstyle{xyp}=[canvas is xy plane at z=0]
\def\tick#1#2{\draw[thick] (#1) ++ (#2:0.12) --++ (#2-180:0.24)}
\def\N{100}


\begin{document}

    \makerappeltitle
    \thispagestyle{firstpage}
    
    Un nombre complexe peut s'écrire sous la forme $z = x+jy$ où $x$ et $y$ correspondent, respectivement, à la partie réelle et imaginaire de $z$. Les électriciens utilisent généralement $j$ au lieu de $i$ pour désigner la partie imaginaire. Le nombre imaginaire $j$ est tel que $j^2 = -1$. On peut représenter $z$ dans le plan complexe en utilisant l’axe des abscisses pour la partie réelle et l’axe des ordonnées pour la partie imaginaire, comme sur la \autoref{fig:complex_plane}. Les différentes images sont reprises du blog \href{https://tikz.net/complex/}{tikz.net}.
    
    \begin{figure}[H]
        \centering
        \begin{tikzpicture}
            \def\xmax{2.0}
            \def\ymax{1.6}
            \def\R{1.9}
            \def\ang{35}
            \coordinate (O) at (0,0);
            \coordinate (R) at (\ang:\R);
            \coordinate (-R) at (-\ang:\R);
            \coordinate (X) at ({\R*cos(\ang)},0);
            \coordinate (Y) at (0,{\R*sin(\ang)});
            \coordinate (-Y) at (0,{-\R*sin(\ang)});
            \node[fill=mydarkblue,circle,inner sep=0.8] (R') at (R) {};
            \node[fill=mydarkred,circle,inner sep=0.8] (-R') at (-R) {};
            \node[mydarkblue,above right=-2] at (R') {$z=x+jy=re^{j\theta}$};
            \node[mydarkred,below right=-1] at (-R') {$z^*=x-jy=re^{-j\theta}$};
            \draw[dashed,mydarkblue]
            (Y) -- (R') --++ (0,{0.1-\R*sin(\ang)});
            \draw[dashed,mydarkred]
            (-Y) -- (-R') --++ (0,{\R*sin(\ang)-0.45});
            \draw[->,line width=0.9] (-0.65*\xmax,0) -- (\xmax+0.05,0) node[right] {Re};
            \draw[->,line width=0.9] (0,-\ymax) -- (0,\ymax+0.05) node[left] {Im};
            \draw[vector] (O) -- (R') node[pos=0.55,above left=-2] {$r$};
            \draw[vector,myred] (O) -- (-R') node[pos=0.55,below left=-2] {$r$};
            \draw pic[->,"$\theta$",mydarkblue,draw=mydarkblue,angle radius=23,angle eccentricity=1.24]
            {angle = X--O--R};
            \draw pic[<-,"$-\theta$"{right=-1},mydarkred,draw=mydarkred,angle radius=20,angle eccentricity=1]
            {angle = -R--O--X};
            %\tick{X}{90} node[scale=0.9,left=6,below right=-2] {$x = r\cos\theta$};
            \tick{X}{90} node[scale=1,below=-1] {$x$};
            \tick{Y}{ 0} node[mydarkblue,scale=1,left] {$y$}; %r\sin\theta = 
            \tick{-Y}{ 0} node[mydarkred,scale=1,left] {$-y$};
        \end{tikzpicture}
        \caption{Nombre complexe dans le plan complexe.}
        \label{fig:complex_plane}
    \end{figure}
    
    Une autre façon de représenter un nombre complexe est d'utiliser son module, $r$, et son argument, $\theta$. Ainsi, $z$ peut s'écrire $r e^{j \theta}$. Le module de $z$, pouvant aussi s'écrire $|z|$, équivaut à la longueur du vecteur $x+jy$, c.-à-d. $|z|=r=\sqrt{x^2 + y^2}$. L'argument de $z$, parfois noté $\angle z$, quant à lui, est l'angle que forme le vecteur avec l'axe des abscisses, tel que $\angle z = \theta = \arctan(y/x)$. Le complexe conjugué de $z$, noté $z^*$, s'obtient en prenant l'opposé de la partie imaginaire de $z$ : $z^*=x - j y$.
    
    \paragraph{Propriétés}
    
    \begin{itemize}
        \item Formule d'Euler : $e^{\pm j a x} = \cos(ax) \pm j \sin(a x)$, pour tous réels $a$ et $x$, voir \autoref{fig:euler}.
        \item Trigonométrie : $\cos(a x) = \frac{e^{ j a x} + e^{-ja x}}{2}$ et $\sin(a x) = \frac{e^{ ja x} - e^{-ja x}}{2j}$.
        \item Inverse : $\frac{1}{j} = \frac{1}{j} \frac{j}{j} = \frac{j}{-1} = -j$.
    \end{itemize}
    
    \begin{figure}[H]
        \centering
        % VECTOR ROTATION
        \def\xmax{2.7}
        \def\ymax{2.7}
        \def\R{2.3}
        \def\ang{28}
        \def\dang{35}
        
        
        % COMPLEX ROTATION
        \begin{tikzpicture}
            \coordinate (O) at (0,0);
            \coordinate (R) at (\ang:\R);
            \coordinate (Q) at (\ang+\dang:\R);
            \coordinate (X) at ({\R*cos(\ang)},0);
            \coordinate (Y) at (0,{\R*sin(\ang)});
            %\draw[myblue] (O) circle (0.995*\R);
            \draw[myblue] (-10:\R) arc (-10:100:\R);
            \node[fill=myred,circle,inner sep=0.8] at (R) {};
            \node[fill=myred,circle,inner sep=0.8] at (Q) {};
            \node[mydarkblue,above right=0] at (R) {$z=re^{j\theta}$};
            \node[mydarkblue,left=2,above right=0] at (Q) {$ze^{j\phi}=re^{j(\theta+\phi)}$};
            \draw[dashed,mydarkblue]
            (Y) -- (R) --++ (0,{0.1-\R*sin(\ang)});
            \draw[->,line width=0.9] (-0.1*\xmax,0) -- (\xmax+0.05,0) node[right] {Re};
            \draw[->,line width=0.9] (0,-0.1*\ymax) -- (0,\ymax+0.05) node[left] {Im};
            \draw[vector] (O) -- (R) node[pos=0.65,above left=-3] {$r$};
            \draw[vector] (O) -- (Q) node[pos=0.65,above left=-3] {$r$};
            \draw pic[->,"$\theta$",mydarkblue,draw=mydarkblue,angle radius=24,angle eccentricity=1.25]
            {angle = X--O--R};
            \draw pic[->,"$\phi$",mydarkblue,draw=mydarkblue,angle radius=20,angle eccentricity=1.35]
            {angle = R--O--Q};
            \tick{0,\R+0.015}{0};
            \tick{\R+0.015,0}{90};
            \tick{X}{90} node[scale=0.9,left=14,below=-1] {$r\cos\theta$};
            \tick{Y}{ 0} node[scale=0.9,below=1,left=-2] {$r\sin\theta$};
        \end{tikzpicture}
        \caption{$z$ dans le plan complexe, avec $ax=\theta$.}
        \label{fig:euler}
    \end{figure}

    Dans beaucoup d'applications électriques, l'argument de $z$ varie au cours du temps (pensez au courant alternatif que vous trouvez dans toutes les prises) et il est commun d'écrire alors $z=A e^{j\omega t}$, avec $A$ l'amplitude du signal en \si{\volt} et $\omega=2 \pi f$ la pulsation en \si{\radian\per\second}. Une visualisation en 3 dimensions d'un nombre complexe oscillant en fonction du temps est faite sur la \autoref{fig:3d}.
    
    \pagebreak
    \pagestyle{rappelnextpages}
    
    \begin{figure}[H]
        \centering
        % COMPLEX OSCILLATOR 3D
\def\xang{-13}
\def\zang{45}
\begin{tikzpicture}[x=(\xang:0.9), y=(90:0.9), z=(\zang:1.1)]
  \message{^^JSynthesis 3D}
  \def\xmax{8.8}         % max x axis
  \def\ymin{-1.5}        % min y axis
  \def\ymax{1.6}         % max y axis
  \def\zmax{1.6}         % max z axis
  \def\xf{1.17*\xmax}    % x position frequency axis
  \def\A{(0.70*\ymax)}   % amplitude
  \def\T{(0.335*\xmax)}  % period
  \def\w{\zmax/11.2}     % spacing components
  \def\ang{47}           % angle
  \def\s{\ang/360*\T}    % time component
  \def\x{\A*cos(\ang)}   % real component
  \def\y{\A*sin(\ang)}   % imaginary component
  
  % COMPLEX PLANE
  \begin{scope}[shift={(-1.6*\zmax,0,0)}]
    \draw[black,fill=white,opacity=0.3,yzp]
      (-1.25*\zmax,-1.25*\ymax) rectangle (1.4*\zmax,1.25*\ymax);
    \draw[->,thick] (0,\ymin,0) -- (0,\ymax+0.02,0)
      node[pos=1,left=0,yzp] {Im};
    \draw[->,thick] (0,0,-\zmax) -- (0,0,\zmax+0.02)
      node[right=1,below=0,yzp] {Re} coordinate (X);
    %\node[scale=1,yzp] at (0,-\ymax,0) {Complex plane};
    \draw[xline,yzp] (0,0) circle(0.991*\A) coordinate (O);
    \fill[myred,yzp] (\ang:{\A}) circle(0.07) coordinate(P);
    \node[mydarkblue,above=3,right=-5,yzp,scale=0.8] at (P) {$z(t)=Ae^{i\omega t}$};
    \draw[vector,thick,yzp] (0,0) -- (\ang:{\A-0.03}) coordinate (R);
    \draw pic[-{>[flex'=1]},draw=mydarkblue,angle radius=14,angle eccentricity=1,
              "$\omega t$"{above=0,right=-0.5,yslant=0.69,scale=0.8},mydarkblue,yzp]
      {angle = X--O--R};
    \tick{0,0,{\A}}{90};
    \tick{0,0,{-\A}}{90};
    \tick{0,{\A},0}{\zang};
    \tick{0,{-\A},0}{\zang};
  \end{scope}
  
  % IMAGINARY
  \begin{scope}[shift={(0,0,1.9*\zmax)}]
    \draw[black,fill=white,opacity=0.3,xyp]
      (-0.5*\ymax,-1.2*\ymax) rectangle (1.10*\xmax,1.25*\ymax);
    \draw[->,thick] (-0.3*\ymax,0,0) -- (\xmax,0,0)
      node[below right=-2,xyp] {$t$ [s]};
    \draw[->,thick] (0,\ymin,0) -- (0,\ymax,0)
      node[left,xyp] {Im};
    \draw[xline,samples=\N,smooth,variable=\t,domain=-0.05*\T:0.95*\xmax]
      plot(\t,{\A*sin(360/\T*\t)},0);
    %\node[below=0,xyp] at (0.4*\xmax,-\ymax,0) {Imaginary component};
    \fill[myred,xyp] ({\s},{\y}) circle(0.07) coordinate(I);
    \draw[vector,thick,xyp] ({\s},0) --++ (0,{\y-0.03});
    \tick{0,{\A},0}{180};
    \tick{0,{-\A},0}{180};
    \tick{{\s},0,0}{90} node[right=0,below=-1,xyp] {$\omega t$};
    \tick{{\T},0,0}{90} node[right=0,below,xyp] {\contour{white}{$T$}};
    \tick{{2*\T},0,0}{90} node[right=0,below,xyp] {\contour{white}{$2T$}};
    \node[mydarkblue,below=0,xyp] at (0.4*\xmax,1.15*\ymax,0) {$y(t)=A\sin(\omega t)$};
  \end{scope}
  
  % REAL
  \begin{scope}[shift={(0,-1.8*\zmax,0)}]
    \draw[black,fill=white,opacity=0.3,xzp]
      (-0.5*\ymax,-1.4*\ymax) rectangle (1.10*\xmax,1.25*\ymax);
    \draw[->,thick] (-0.3*\ymax,0,0) -- (\xmax,0,0)
      node[below right=-1,xzp] {$t$ [s]};
    \draw[->,thick] (0,0,-\zmax) -- (0,0,\zmax)
      node[left=-1,xzp] {Re};
    \draw[xline,samples=\N,smooth,variable=\t,domain=-0.05*\T:0.95*\xmax]
      plot(\t,0,{\A*cos(360/\T*\t)});
    %\node[below=0,xzp] at (0.4*\xmax,-\ymax,0) {Real component};
    \fill[myred,xzp] ({\s},{\x}) circle(0.07) coordinate(R);
    \draw[vector,thick,xzp] ({\s},0) --++ (0,{\x-0.03});
    \tick{0,0,{\A}}{180};
    \tick{0,0,{-\A}}{180};
    \tick{{\s},0,0}{\zang} node[below=-1,xzp] {$\omega t$};
    \tick{{\T},0,0}{\zang} node[below,xzp] {$T$};
    \tick{{2*\T},0,0}{\zang} node[below,xzp] {$2T$};
    \node[mydarkblue,above=0,xzp] at (0.3*\xmax,-\ymax,0) {$x(t)=A\cos(\omega t)$};
  \end{scope}
  
  % COMPONENTS
  \draw[myred!80!black,dashed]
    (P) -- ({\s},{\y},{\x})
    (I) -- ({\s},{\y},{\x+0.05})
    ({\s},{\y-0.06},{\x}) -- (R);
  \draw[->,black,thick] (-0.1*\ymax,0,0) -- (\xmax,0,0) node[below right=-2] {$t$ [s]};
  \draw[->,black,thick] (0,\ymin,0,0) -- (0,\ymax+0.02,0) node[above] {Im};
  \draw[->,black,thick] (0,0,-\zmax) -- (0,0,\zmax+0.02) node[right=1,below=3] {Re};
  \foreach \i [evaluate={\tmin=max(-0.05*\T,(\i-0.05)*\T); \tmax=min(0.95*\xmax,(\i+1)*\T);}] in {0,...,2}{
    %\draw[white,line width=1.2] (\tmin,0,0) -- (\tmax,0,0);
    \draw[thick] (\tmin,0,0) -- (\tmax,0,0);
    \draw[xline,samples=0.4*\N,smooth,variable=\t]
      plot[domain=\tmin:\tmax](\t,{\A*sin(360/\T*\t)},{\A*cos(360/\T*\t)});
  }
  \draw[thick] (0,0,{0.9*\A}) -- (0,0,{\A});
  \fill[myred] ({\s},{\y},{\x}) circle(0.07) coordinate(Z);
  \draw[vector,thick] ({\s},0,0) --++ (0,{\y-0.03},{\x-0.03});
  \draw[xline,samples=0.3*\N,smooth,variable=\t,domain=\s+0.03:\s+0.4*\T,line cap=round]
    plot(\t,{\A*sin(360/\T*\t)},{\A*cos(360/\T*\t)});
  \tick{{\T},0,0}{90};
  \tick{{2*\T},0,0}{90};
  \tick{0,0,{\A}}{90};
  \tick{0,0,{-\A}}{90};
  \tick{0,{\A},0}{\zang};
  \tick{0,{-\A},0}{\zang};
  \draw[myred!80!black,dashed]
    ({\s},{\y-0.06},{\x}) --++ (0,-0.2*\ymax,0);
  
\end{tikzpicture}
        \caption{Oscillateur complexe vu en 3D.}
        \label{fig:3d}
    \end{figure}

\end{document}